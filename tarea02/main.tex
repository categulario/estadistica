\documentclass[11pt,letterpaper,reqno]{article}

\textwidth      =  6in
\textheight     =  8.25in
\oddsidemargin  =  18pt
\evensidemargin =  18pt
\topmargin      =  0.00in

\usepackage[utf8]{inputenc}
\usepackage[spanish]{babel}
\usepackage{amsmath}
\usepackage{amsfonts}
\usepackage{amssymb}
\usepackage{graphicx}
\usepackage{minted}
\graphicspath{{img/}}

\title{Implementación y Simulaciones del Modelo $k$-nn}
\author{Abraham Toriz\\ Jesús Mejía\\ Luis Cortés\\ Roberto Saucedo}

\begin{document}
\maketitle
\begin{abstract}
Este trabajo se da la implementación  de del modelo $k$-nn, después se hacen simulaciones y se hacen comentarios de los resultados obtenidos.
\end{abstract}

\section{Implementación en Python}
Sea \verb|data = |$\{(X_i, Y_i): i=1,\;2,\ldots,\;N,\; Y_i=1 \text{ ó } Y_i=2\}$ el conjunto de $N$ observaciones y \verb|x| un dato que queremos clasificar. En estos ejemplos, usaremos la métrica eucliana y los cálculos los hará la función \verb|metric(x, y)|.\\

Primero debemos calcular las todas distancias entre \verb|x| y los $X_i$ en \verb|data|
\begin{minted}{python}
    distances = []
    for dat in data:
        distances.append((metric(dat[0], x), dat[1]))
\end{minted}

Una vez guardadas las distancias, se ordenan ascendentemente, es decir:
\begin{minted}{python}
    distances.sort(key=lambda x: x[0])
\end{minted}
Luego se encuentra el número de elementos de la clase uno y de la clase dos para los $k$ vecinos más cercanos:
\begin{minted}{python}
    classes = {}
    for item in distances[:k]:
        if item[1] in classes:
            classes[item[1]] += 1
        else:
            classes[item[1]] = 1
\end{minted}
Finalmente se selecciona el mayor número de elementos que tienen la mis clase:
\begin{minted}{python}
    classes = list(classes.items())
    classes.sort(key=lambda x: x[1])
    return classes[-1][0]
\end{minted}
\end{document}